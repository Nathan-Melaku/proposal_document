\documentclass[../main/main.tex]{subfiles}


\begin{document}
	
	\begin{abstract}
	\noindent Steganographic algorithms are mainly evaluated by their security. Traditional steganographic frameworks used different embedding algorithms to achieve this goal. This means, embedding the secret message directly to a cover image. Nonetheless, with the development of sophisticated machine learning based steganalysis algorithms even the slightest modifications can be detected. This has triggered a lot of researchers to pursue coverless steganography. Image sythesis with generative adversarial networks (\gls{GAN}) is one of the proposed solutions. However it has suffered drawbacks in convergence speed, training stability and realistic image generation. This research is about designing and implementing a generative steganography framework using the state of the art \gls{GAN} with the main goal of improving the quality of the generated stego images without compromising the security.  
	
	\end{abstract}

	\keywords{Generative Steganography, GAN, Security}

	
\end{document}