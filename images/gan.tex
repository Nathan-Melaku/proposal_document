\documentclass[../main/main.tex]{subfiles}

\begin{document}
	\begin{figure}[h]
		
		\centering
		\begin{tikzpicture}
		[node distance = 0.5cm, auto,font=\footnotesize,
		% STYLES
		every node/.style={node distance=2cm},
		% The comment style is used to describe the characteristics of each force
		comment/.style={rectangle, inner sep= 5pt, text width=2cm, node distance=0.25cm, font=\scriptsize\sffamily},
		% The force style is used to draw the forces' name
		force/.style={rectangle, draw, fill=black!10, inner sep=5pt, text width=2.5cm, text badly centered, minimum height=1.2cm, font=\bfseries\footnotesize\sffamily}] 
		
		% Draw forces
		\node [force] (generator) {Generator};
		\node [force, left=1cm of generator] (Inputvector) {Input Vector $z$};
		\node [force, right=1cm of generator] (discriminator) {Discriminator};
		\node[force, above of=generator, right=1cm of generator](realdata){Real Image Data Set};
		\node[force, right=1cm of discriminator](Real){Real};
		\node[force, above of=Real](Fake){Fake};
		
		% Draw the links between forces
		\path[->,thick] 
		(Inputvector) edge (generator)
		(generator) edge (discriminator)
		(realdata) edge (discriminator)
		(discriminator) edge (Fake)
		(discriminator) edge (Real);
		
		\end{tikzpicture} 
		\caption{A Generative Adversarial Network}
		\label{fig:1gan}
	\end{figure}
\end{document}