\documentclass[../main/main.tex]{subfiles}

\begin{document}
	
	\section{Scope}
	This section clearly specifies what this research will focus on. The scope of this research is shown as follows.
	
	\begin{itemize}
		\item This research does not focus on robustness of generative steganographic frameworks. Robustness in this scenario is defined in section \ref{reqStego}. Improving or analyzing robustness is outside of the scope of this research.
		
		\item In the classic paper \shortcite{simmons1984prisoners} the warden acts as an active adversary. He can change or modify the contents of the message to trick Alice and Bob. However in all of generative steganographic frameworks proposed till now passive adversary is assumed. \shortcite{Hu2018}, \shortcite{Zhang2019}, \shortcite{Ke}. This research will also assume passive adversary. In order to address the problem with active adversary, steganographic frameworks complemented with  cryptographic frameworks would provide a good solution. Nonetheless, this would be an area for further study.
		
		\item Increasing payload capacity of generative steganography is outside the scope of this research. However payload capacity of the proposed framework in this research must be comparable with state of the art frameworks. 
		
		\item Out of the three requirements of steganography shown in section \ref{reqStego} this research focus on Improving the first one, namely \textbf{Imperceptibility}.
	\end{itemize}
\end{document}