\documentclass[../main/main.tex]{subfiles}

\begin{document}
	
	\section{Statement Of Problem} \label{statement}
	Generative Steganography has better defense against steganalysis tools. The main reason behind this strength is  synthesis of cover images and the lack of modification. Despite this useful advantage, Generative steganography is at its infancy. The main drawbacks are instability of training, slow convergence speed, and inadequately realistic image generation. This is mainly associated to the \gls{GAN} technique used in the framework. As shown in section \ref{Literature} the \gls{GAN} technique mostly used is \gls{DCGAN}, and to lesser extent \gls{ACGAN}, \gls{infoGAN} and \gls{WGAN}. This problem is explicitly mentioned in \cite{Hu2018}, \cite{Zhang2019}, and \cite{Ke}.
	However, \gls{GAN}s have been improved through the past five years. And there are more than 30 variants available at this time. This calls for a better generative steganography framework using a powerful \gls{GAN}.
\end{document}